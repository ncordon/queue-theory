%%%
% Plantilla de Presentación
% Modificación de una plantilla de Latex de LaTeXTemplates para adaptarla
% al castellano y a las necesidades de escribir informática y matemáticas.
%
% Editada por: Mario Román
%
% License:
% CC BY-NC-SA 3.0 (http://creativecommons.org/licenses/by-nc-sa/3.0/)
%%%

%%%%%%%%%%%%%%%%%%%%%%%%%%%%%%%%%%%%%%%%%
% Beamer Presentation
% LaTeX Template
% Version 1.0 (10/11/12)
%
% This template has been downloaded from:
% http://www.LaTeXTemplates.com
%
% License:
% CC BY-NC-SA 3.0 (http://creativecommons.org/licenses/by-nc-sa/3.0/)
%
%%%%%%%%%%%%%%%%%%%%%%%%%%%%%%%%%%%%%%%%%

%----------------------------------------------------------------------------------------
%	PAQUETES Y CONFIGURACIÓN DEL DOCUMENTO
%----------------------------------------------------------------------------------------

\documentclass[8pt]{beamer}
%\geometry{paperwidth=140mm,paperheight=105mm}

%% Configuración de la presentación
\mode<presentation> {
  %%% Selección de estilo
  % The Beamer class comes with a number of default slide themes
  % which change the colors and layouts of slides. Below this is a list
  % of all the themes, uncomment each in turn to see what they look like.

  %\usetheme{default}
  %\usetheme{AnnArbor}
  %\usetheme{Antibes}
  %\usetheme{Bergen}
  %\usetheme{Berkeley}
  %\usetheme{Berlin}
  %\usetheme{Boadilla}
  %\usetheme{CambridgeUS}
  %\usetheme{Copenhagen}
  %\usetheme{Darmstadt}
  \usetheme[compress]{Dresden}
  %\usetheme{Frankfurt}
  %\usetheme{Goettingen}
  %\usetheme{Hannover}
  %\usetheme{Ilmenau}
  %\usetheme{JuanLesPins}
  %\usetheme{Luebeck}
  %\usetheme{Madrid}
  %\usetheme{Malmoe}
  %\usetheme{Marburg}
  %\usetheme{Montpellier}
  %\usetheme{PaloAlto}
  %\usetheme{Pittsburgh}
  %\usetheme{Rochester}
  %\usetheme{Singapore}
  %\usetheme{Szeged}
  %\usetheme{Warsaw}

  %% Selección de color
  % As well as themes, the Beamer class has a number of color themes
  % for any slide theme. Uncomment each of these in turn to see how it
  % changes the colors of your current slide theme.

  %\usecolortheme{albatross}
  %\usecolortheme{beaver}
  %\usecolortheme{beetle}
  %\usecolortheme{crane}
  %\usecolortheme{dolphin}
  %\usecolortheme{dove}
  %\usecolortheme{fly}
  \usecolortheme{lily}
  %\usecolortheme{orchid}
  %\usecolortheme{rose}
  %\usecolortheme{seagull}
  %\usecolortheme{seahorse}
  %\usecolortheme{whale}
  %\usecolortheme{wolverine}

  %% Configuración del pie de línea
  %\setbeamertemplate{footline} % To remove the footer line in all slides uncomment this line
  %\setbeamertemplate{footline}[page number] % To replace the footer line in all slides with a simple slide count uncomment this line
  %\setbeamertemplate{navigation symbols}{} % To remove the navigation symbols from the bottom of all slides uncomment this line
}

\setbeamertemplate{section in toc}[sections numbered]
\setbeamertemplate{subsection in toc}[subsections numbered]

%% Fuentes de tamaño arbitrario
\usepackage{lmodern}

\usepackage{listings}
\lstset{language=C++,
                basicstyle=\ttfamily,
                keywordstyle=\color{blue}\ttfamily,
                stringstyle=\color{red}\ttfamily,
                commentstyle=\color{green}\ttfamily,
                morecomment=[l][\color{magenta}]{\#}
}

%% Gráficos
\usepackage{graphicx} % Allows including images
\usepackage{booktabs} % Allows the use of \toprule, \midrule and \bottomrule in tables

\newcommand{\hlink}[2]{{\color{blue}\href{#1}{#2}}}
%%% Castellano.
% noquoting: Permite uso de comillas no españolas.
% lcroman: Permite la enumeración con numerales romanos en minúscula.
% fontenc: Usa la fuente completa para que pueda copiarse correctamente del pdf.
\usepackage[spanish,es-noquoting,es-lcroman]{babel}
\usepackage[utf8]{inputenc}
\usepackage[T1]{fontenc}
\selectlanguage{spanish}

%% Justificación del texto
\usepackage{ragged2e}
\usepackage{etoolbox}
\addtobeamertemplate{block begin}{}{\justifying}
\apptocmd{\frame}{\justifying}{}{}
%\apptocmd{\column}{\justifying}{}{}


\usepackage{tikz}
\usetikzlibrary{arrows}

%----------------------------------------------------------------------------------------
%	TÍTULO
%----------------------------------------------------------------------------------------

\title[]{Procesos estocásticos: } % The short title appears at the bottom of every slide, the full title is only on the title page
\subtitle[]{Modelos de colas}

\author[Marta Andrés, Ignacio Cordón, Bartolomé Ortiz] % Your name
{\texorpdfstring{
      \centering
      Marta Andrés, Ignacio Cordón, Bartolomé Ortiz\\
      \href{https://www.github.com/ncordon/queue-theory}{www.github.com/ncordon/queue-theory}
}{Marta Andrés, Ignacio Cordón, Bartolomé Ortiz}}

\institute[UGR] % Your institution as it will appear on the bottom of every slide, may be shorthand to save space
{
  Universidad de Granada \\ % Your institution for the title page
  \medskip
  %\textit{autor@ugr.correo.es} % Your email address
}
\date{\today} % Date, can be changed to a custom date

% \subtitle{Y a la programación funcional}   % Subtítulo
% \author[@pbaeyens \and @M42]    % Autores (tex.stackexchange.com/questions/63259)
% {\texorpdfstring{
%     \begin{columns}
%       \column{.45\linewidth}
%       \centering
%       Pablo Baeyens\\
%       \href{http://www.github.com/pbaeyens}{@pbaeyens}
%       \column{.45\linewidth}
%       \centering
%       Mario Román\\
%       \href{http://www.github.com/M42}{@M42}
%     \end{columns}
% }{Pablo Baeyens \and Mario Román}}
% \date{OSL 2015}

\usepackage{scalerel}
\newcommand*\bigcdot{\mathpalette\bigcdot@{.5}}

\DeclareMathOperator*{\Bigcdot}{\scalerel*{\cdot}{\bigodot}}

\begin{document}

%% Diapositiva de título.
\begin{frame}
\titlepage % Print the title page as the first slide
\end{frame}

%% Diapositiva de contenidos.
% Throughout your presentation, if you choose to use \section{} and \subsection{} commands,
% these will automatically be printed on this slide as an overview of your presentation
\begin{frame}
  \frametitle{Contenidos} % Table of contents slide, comment this block out to remove it
  \tableofcontents
\end{frame}



%----------------------------------------------------------------------------------------
%	PRESENTACIÓN
%----------------------------------------------------------------------------------------

  \section{Introducción}
  \subsection{Distribución exponencial}
  \begin{frame}\frametitle{Definiciones}
    \begin{block}{Distribución exponencial}
      \[X \sim exp(\lambda),\lambda > 0 \Leftrightarrow F(x) = \left\{\begin{array}{ll}
      1- e^{-\lambda x} ,& x\ge 0 \\
      0 ,& x\le 0
      \end{array}\right.\]
    \end{block}

    \begin{block}{Propiedad de Markov}
      $X \ge 0$ tiene la propiedad de Markov si:
      \[P[X \ge t+h \mid X \ge t] = P[X \ge h] \qquad \forall t,h \in \mathbb{R}_0^{+}\]
      Equivalentemente:
      \[P[X < t+h \mid X \ge t] = P[X < h] = P[ 0 \le X < h] \qquad \forall t,h \in \mathbb{R}_0^{+}\]
    \end{block}
  \end{frame}
  \begin{frame}\frametitle{Propiedades}
    \begin{enumerate}
    \item
      $X \sim exp(\lambda) \Rightarrow X$ tiene la propiedad de Markov
    \item
      $X\ge 0$ continua con la propiedad de Markov $\Rightarrow X \sim exp(\lambda)$
    \end{enumerate}
  \end{frame}


  \subsection{Procesos de Poisson}
  \begin{frame}\frametitle{Notación}
    \begin{block}{Notación $o$}
    Una función $f$ se dice $o(h)$ (formalmente $f\in o(h)$) y lo notamos $f=o(h)$ si se verifica:
    \[\lim_{h\rightarrow 0} \frac{f(h)}{h} = 0\]
    \end{block}
    \begin{block}{Linealidad}
      Dados $c_1, \ldots c_n \in \mathbb{R}$, $f_1, \ldots f_n \in o(h)$, entonces $\sum_{i=1}^{n} c_i f_i = o(h)$
    \end{block}
  \end{frame}

  \begin{frame}\frametitle{Procesos de conteo}
    Sea $\{N_t\}_{t\ge 0}$ proceso estocástico discreto. Se dice que es proceso de conteo si se verifica:
    \begin{enumerate}
    \item No negatividad: $N_t \in \mathbb{N}\cup\{0\}, \quad \forall t\ge 0$. Además: $N_0=0$
    \item Monotonía: $N_s \le N_t, \quad \forall s \le t$
    \end{enumerate}

    $N_t$ indica el número de eventos que han ocurrido en el intervalo $[0,t]$. Por tanto $N_t- N_s$, con $t\ge s$
    indica el número de eventos que han ocurrido en $]s,t]$.
  \end{frame}

  \begin{frame}\frametitle{Procesos de Poisson}
    Un proceso de conteo $\{N_t\}_{t\ge 0}$ se dice que es de Poisson de parámetro $\lambda > 0$ si se verifica:

    \begin{enumerate}
    \item El proceso tiene incrementos independientes: dados $0 \le t_1 < \ldots < t_n$, se verifica que
      las variables $N_{t_1}, N_{t_2} - N_{t_1}, \ldots, N_{t_n}- N_{t_{n-1}}$ son independientes. Esto es, el número de eventos
      que se producen en intervalos disjuntos es independiente.
    \item El proceso tiene incrementos estacionarios: $dist(N_{t+h} - N_t)$ es la misma para cualesquiera
      $t\ge 0, h\ge 0$.
    \item $P[N_h = 1] = \lambda h + o(h)$, es decir, la probabilidad de que ocurra un evento en un intervalo de
      tiempo de longitud $h$ es casi proporcional a $h$, salvo por un término despreciable en comparación con dicho $h$, para
      $h$ suficientemente pequeño.
    \item $P[N_h \ge 2] = o(h)$.
    \end{enumerate}

    Se deduce que:
    \[P[N_h = 0] = 1 - P[N_h=1] - P[N_h \ge 2] = 1 -\lambda h - o(h)\]
  \end{frame}

  \begin{frame}\frametitle{Propiedades}
    La mayoría de modelos de colas asumen una distribución exponencial para tiempos entre llegadas y tiempos 
de servicio, o equivalentemente una distribución de Poisson para frecuencias de llegada y servicio.

    \begin{enumerate}
    \item
      Sea $\{N_t\}_{t\ge 0}$ un proceso de Poisson de parámetro $\lambda > 0$. Entonces la variable aleatoria $Y_t = N_t - N_0 = N_t$ que
      describe el número de eventos en cualquier intervalo de longitud $t > 0$ tiene una distribución de Poisson de parámetro
      $\lambda t$:

      \[P[Y_t = k] = P[N_t = k] = e^{-\lambda t} \frac{(\lambda t)^k}{k!}, \quad k\ge 0\]

    \item
      Sea $\{N_t\}_{t\ge 0}$ proceso de conteo. Sea $\{t_n\}_{n\ge 1}$ sucesión real estrictamente creciente y positiva, 
      que representa los tiempos de eventos, es decir \[t_n = \min \{t \ge 0: N_t = n\}\]

      LLamando $\tau_1= t_1, \tau_{n+1} = t_{n+1} - t_{n}, \quad \forall n\in \mathbb{N}$ tiempos entre llegadas. Entonces equivalen:

      \begin{itemize}
      \item $\{N_t\}_{t\ge 0}$ es proceso de Poisson.
      \item Los tiempos entre llegadas $\{\tau_n\}$ son variables exponenciales i.i.d. de media $\frac{1}{\lambda}$, esto es,
        $\tau_n \sim exp(\lambda)$.
      \end{itemize}
    \item
      Sea $\{N_t\}_{t\ge 0}$ proceso de Poisson donde un evento ha tenido lugar en $[0,t]$. Entonces siendo $Y$ la variable
      describiendo el tiempo de ocurrencia de dicho evento en el intervalo $[0,t]$, se tiene $Y \sim U([0,t])$.

    \end{enumerate}
  \end{frame}

  \subsection{Procesos de nacimiento y muerte}
  \begin{frame}\frametitle{Procesos de nacimiento y muerte}
    %% El parámetro $\lambda$ de un proceso de Poisson $\{N_t\}_{t\ge 0}$ puede ser visto como una tasa de nacimiento, ya que la probabilidad
    %% de que ocurra un evento en un intervalo de longitud $h > 0$ es $P[N_h-N_0=1] = \lambda h e^{-\lambda h} = \lambda h + o(h)$.
    %% Cuando suponemos que el parámetro no es constante, sino que depende de $n$ (cantidad de eventos que se han producido hasta el momento), esto
    %% es $\lambda_n$, entonces la cantidad de nacimientos (eventos producidos) en un intervalo de longitud $h$ es $\lambda_n h + o(h)$. Si además
    %% establecemos que se pueden producir muertes con una tasa $\mu_n$, donde la probabilidad de que se produzca una muerte en un intervalo de longitud
    %% $h$ es $\mu_n h + o(h)$ tenemos un proceso de nacimiento y muerte.
    Sea una cadena de Markov $\{N_t\}_{t\ge 0}$ con espacio de estados $\mathbb{N}\cup \{0\}$, donde el espacio de estados representa el número de individuos
    de un sistema (población). Entonces $\{N_t\}_{t\ge 0}$ se dice proceso de nacimiento y muerte si existen tasas no negativas de nacimiento y muerte,
    $\{\lambda_n\}_{\mathbb{N}\cup \{0\}}$ y $\{\mu_n\}_{\mathbb{N}\cup \{0\}}$ y se verifica:
    \begin{enumerate}
    \item La población puede aumentar o decrecer únicamente de uno en uno.
    \item Si el sistema está en estado $n\ge 0$ entonces el tiempo hasta que el sistema está en estado $n+1 \ge 0$ es una variable aleatoria exponencial
      de paŕámetro $\lambda_n$.
    \item Si el sistema está en estado $n\ge 1$ entonces el tiempo hasta que el sistema está en estado $n-1 \ge 0$ es una variable aleatoria exponencial
      de paŕámetro $\mu_n$.
    \item Cualesquiera dos transiciones son independientes.
    \end{enumerate}
  \end{frame}

  \begin{frame}\frametitle{Distribución límite}
  Llamando $P_n(t) = P[N_t = n]$ se verifica:
    \[P_n(t+h) = [1-\lambda_n h -\mu_n h] P_n(t) + \lambda_{n-1} h P_{n-1}(t) + \mu_{n+1} h P_{n+1}(t) + o(h)\]
    
    \begin{equation}
     \frac{\partial P_n(t)}{\partial t} = -(\lambda_n + \mu_n) P_n(t) + \lambda_{n-1}P_{n-1}(t) + \mu_{n+1}P_{n+1}(t)
     \label{eq:recpn(t)}
    \end{equation}

    Supongamos en lo que sigue que existe la distribución límite (entonces existiría la estacionaria), esto es:
    
    \begin{equation}
     \lim_{t\rightarrow \infty}\{P_n(t)\} = p_n     
    \end{equation}

    y haciendo $t\rightarrow \infty$ en \eqref{eq:recpn(t)},
	
	\begin{align*}
	0 &= \lambda_{n-1} p_{n-1} + \mu_{n+1} p_{n+1} - (\lambda_n + \mu_n) p_n, \quad n\ge 1\\
	0 &= \mu_1 p_1 -\lambda_0 p_0, \quad n=0
	\end{align*}
	
    Justificamos que la parte izquierda se hace cero: habríamos llegado a que $\frac{\partial P_n(t)}{\partial t}$ tiende a una constante $C$,
    pero la $P_n$ tiende a $p_n$, luego no es difícil demostrar que $C=0$.
  \end{frame}

  \begin{frame}
  Por inducción es fácil probar:
  
  \begin{equation}
    p_n = p_0 \prod_{i=1}^n \frac{\lambda_{i-1}}{\mu_i}, \quad n\ge 1
    \label{pn}
  \end{equation}
  
   \begin{equation}
    1 = \sum_{n=0}^{\infty} p_n = p_0 \underbrace{\left(1 + \sum_{n=1}^{\infty} \prod_{i=1}^n \frac{\lambda_{i-1}}{\mu_i} \right)}_{Z}
    \label{eq:stabseries}
   \end{equation}

   Que la serie $Z$ sea convergente es condición necesaria para que exista la distribución límite. 
   De hecho, también es condición suficiente (se puede probar por teorema de ergodicidad). Si dicha distribución existe se tendría que la probabilidad de
   que en equilibrio haya $0$ peticiones en la cola será:
   
   \begin{equation}
    p_0 = Z^{-1}
    \label{p0}
   \end{equation}

  \end{frame}

  
  
  \section{El modelo de colas}
  \subsection{Definición}
  \begin{frame}\frametitle{Ilustración de las variables clave}
    \begin{figure}[h]
      \centering
      \begin{tikzpicture}
        \draw (20,5.5) rectangle (22,6);
        \node [above] at (21,5.5) {Servidor 1};
        \draw (20,4.5) rectangle (22,5);
        \node [above] at (21,4.5) {Servidor 2};
        \draw [dotted, ultra thick] (21,4.15) -- (21,3.85);
        \draw (20,3) rectangle (22,3.5);
        \node [above, blue] at (21,3) {Servidor $c$};

        \draw [fill] (19.5,5.75) circle [radius=0.1];
        \draw [fill=white] (19.5,4.75) circle [radius=0.1];
        \draw [fill] (19.5,3.25) circle [radius=0.1];
        \draw [dashed, blue] (19.25,3) rectangle (19.75,6);
        \node [below, blue] at (19.5,3) {$N_{S,t}$};

        \draw [latex-latex, blue] (19.25,6.5) -- (22,6.5);
        \draw [blue] (19.25,6) --(19.25,6.6);
        \draw [blue] (22,6) --(22,6.6);
        \node [above, blue] at (20.7,6.5) {$S$};

        \draw (22.15,6) -- (22.5,6) -- (22.5,3) -- (22.15,3);
        \draw [-latex] (22.5,4.5) -- (23.5,4.5);

        \draw (19,6) -- (18.65,6) -- (18.65,3) -- (19,3);
        \draw [-latex] (18,4.5) -- (18.65,4.5);
        \draw [fill] (18,4.5) circle [radius=0.1];
        \draw [fill] (17.6,4.5) circle [radius=0.1];
        \draw [dotted, ultra thick] (17,4.5) -- (17.3,4.5);
        \draw [fill] (16.7,4.5) circle [radius=0.1];
        \draw (16,4.5) -- (16.6,4.5);
        \draw [blue] (16.6,4.3) -- (16.6,4.1) -- (18.1,4.1) -- (18.1,4.3);
        \node [below, blue] at (17.35,4.1) {$N_{Q,t}$};

        \draw [latex-latex, blue] (16.05,5) -- (18.6,5);
        \node [above, blue] at (17.34,5) {$Q$};

        \draw (16,2) rectangle (23,7.5);
        \node [above] at (19.5,7.5) {Modelo de cola};

        \draw [-latex] (13.6,4.5) -- (16,4.5);
        \draw [fill] (14.25,4.5) circle [radius=0.1];
        \draw [fill] (15.25,4.5) circle [radius=0.1];

        \draw [latex-latex, blue] (14.25,5) -- (15.25,5);
        \draw [blue] (14.25,4.65) --(14.25,5.1);
        \draw [blue] (15.25,4.65) --(15.25,5.1);
        \node [above, blue] at (14.75,5) {$\tau$};
      \end{tikzpicture}
    \end{figure}

  \end{frame}
  \begin{frame}\frametitle{Definición de las variables clave}
    \begin{itemize}
    \item [$c$]
      Número (fijo) de servidores o canales en el sistema, $c\in \mathbb{N} \cup \{+\infty\}$
    \item [$\tau$]
      Variable aleatoria que describe el tiempo entre llegadas (de clientes).
    \item [$S$]
      Variable aleatoria que describe el tiempo de servicio.
    \item [$Q$]
      Variable aleatoria que describe el tiempo que espera un cliente en la cola.
    \item [$N_{S,t}$]
      Variable aleatoria que describe el número de clientes que están siendo servidos en el instante $t$.
    \item [$N_{Q,t}$]
      Variable aleatoria que describe el número de clientes en la cola (esperando a ser servidos) en el instante $t$.
    \end{itemize}
  \end{frame}

  \begin{frame}\frametitle{Otras variables relevantes}
    \begin{itemize}
    \item [$\lambda$]
      Frecuencia o tasa media de llegadas de clientes al sistema: $\lambda = 1/E[\tau]$.
    \item [$\mu$]
      Frecuencia o tasa media de servicio de los servidores del sistema: $\mu = 1/E[S]$.
    \item [$\rho$]
      Aprovechamiento de los servidores, esto es, la proporción de tiempo que los servidores están trabajando: $\rho = \frac{\lambda}{c\mu}$.
    \item [$W_S$]
      Tiempo medio que está siendo servido un cliente: $W_S  = E[S]$.
    \item [$W_Q$]
      Tiempo medio que está un cliente en la cola: $W_Q = E[Q]$.
    \item [$W$]
      Tiempo de espera. Variable aleatoria que describe el tiempo total que un cliente está en el sistema
      de encolado: $W = Q+S$.
    \item [$W_W$] Tiempo medio (de espera) que está un cliente en el sistema: $W_W = E[W]$.
    \end{itemize}
    \end{frame}
  \begin{frame}\frametitle{Otras variables relevantes}
    \begin{itemize}
    \item [$N_S$]
      Variable aleatoria que describe el número de clientes siendo servidos con el sistema en equilibrio: $N_S = \lim_{t \rightarrow \infty} N_{s,t}$.
    \item [$L_S$]
      Número medio de clientes siendo servidos con el sistema en equilibrio: \\ $L_S = E[N_S]$.
    \item [$N_Q$]
      Variable aleatoria que describe el número de clientes en la cola con el sistema en equilibrio: $N_Q = \lim_{t \rightarrow \infty} N_{Q,t}$.
    \item [$L_Q$]
      Número medio de clientes en la cola con el sistema en equilibro: $L_Q = E[N_Q]$.
    \item [$N_t$]
      Variable aleatoria que describe el número de clientes en el sistema en el instante $t$: $N_t = N_{Q,t} + N_{S,t}$.
    \item [$N$]
      Variable aleatoria que describe el número de clientes en el sistema con el sistema en equilibrio $N = N_Q + N_S$.
    \item [$L$]
      Número medio de clientes en el sistema en equilibrio: $L = E[N]$.
    \item [$P_n(t)$]
      Probabilidad de que haya $n$ clientes en el sistema en el instante $t$: es la función masa de probabilidad de $N_t$.
    \item [$p_n$]
      Probabilidad de que haya $n$ clientes en el sistema con el sistema en equilibrio: $p_n = \lim_{t \rightarrow \infty} P_n(t)$ es la función masa de probabilidad de $N$.
    \end{itemize}
  \end{frame}

  \subsection{Características}
  \begin{frame}\frametitle{Modelo de llegadas}
    \begin{block}{Definición}
    LLamamos modelo de llegadas a la distribución de los tiempos entre llegadas $\{\tau_n\}$, variables i.i.d.
    \end{block}
    \begin{block}{Definición}

    El modelo de llegadas se dice:
 
    \begin{enumerate}
    \item Estacionario, cuando la distribución no depende del tiempo.
    \item Transitorio, en caso opuesto.
    \end{enumerate}
    \end{block}
    \begin{block}{Definición}

      Para un modelo de llegadas, sea $A(t)$ función de distribución de los $\tau_n$.
      El modelo de llegadas se dice:

      \begin{enumerate}
      \item Aleatorio o de Poisson: si $A(t) = 1 - e^{-\lambda t}$.
      \item Determinístico: si $A(t) = \left\{\begin{array}{ll}
        0 & t<s \\
        1 & t\ge s
      \end{array}\right.$ con $s$ fijo.

      \end{enumerate}
    \end{block}
  \end{frame}
  \begin{frame}\frametitle{Modelo de servicio}
    Se pueden efectuar unas definiciones análogas para modelo determinístico, aleatorio y régimen estacionario y transitorio
    de servicio a las que se han hecho con el modelo de llegadas.

    \begin{block}{Ejemplo: modelo de Poisson de servicio}

      Supongamos que el modelo de colas tiene $k$ servidores o canales para atender peticiones, y el tiempo que 
      tarda cada uno en atender una petición $T_i, i=1,\ldots k$ sigue una distribución exponencial de parámetro $\bar{\mu}$.
      Entonces el tiempo hasta terminar de atender una petición es $T = min\{T_1, \ldots T_n\}$. Por propiedades de la
      distribución exponencial deducimos que $T$ sigue una distribución exponencial de parámetro $\mu = n \bar{\mu}$
      
      La función de distribución de $W_S$ por tanto sería $W_S[t] = P[S\le t] = 1 - e^{-\mu t}$.

    \end{block}


  \end{frame}
  \begin{frame}\frametitle{Capacidad del sistema}
    Sea $S\subseteq \mathbb{N}$ el menor espacio de estados sobre el que está definido el proceso estocástico $\{N_{Q,t}\}_{t\ge 0}$ . 
    Se llama capacidad al tamaño máximo de la cola de peticiones, esto es $M = \max S \in \mathbb{N} \cup \{+\infty\}$. 
    Por tanto no puede tenerse que en ningún tiempo $N_{Q,t}$ tome un valor mayor que $M$, y si llegara una
    petición cuando la cola tiene tamaño $M$, se rechazaría dicha petición.
  \end{frame}
  \begin{frame}\frametitle{Comportamiento de los clientes}
    Podemos considerar que los clientes que llegan a la cola pueden efectuar:

    \begin{itemize}
    \item Oposición: una petición se retira en la llegada (por ejemplo en casos en los que la cola sea muy larga
      y el cliente se abstenga de entrar).
    \item Retirada: la petición entra a la cola, pero tras un tiempo de espera, la deja.
    \item Recolocación: si hay varias colas paralelas, los clientes se pueden cambiar de una cola a otra.
    \item Priorización: algunos clientes pueden tener mayor prioridad que otros al ser servidos.
    \end{itemize}

  \end{frame}
  \begin{frame}\frametitle{Disciplina de la cola}
    Se pueden tener distintas formas de elegir de la cola el siguiente cliente a ser servido. Consideramos las siguientes estrategias:
    \begin{itemize}
    \item FIFO (First in, first out): es la cola tradicional, en la que el cliente que lleva más tiempo esperando en la cola es el siguiente en ser servido.
    \item LIFO (Last in, first out): el cliente que lleva menos tiempo esperando en la cola es el siguiente en ser servido.
    \item SIRO (Service in random order): se escoge un cliente al azar de entre todos los de la cola, con probabilidad uniforme.
    \item PRI (Priority  service): los clientes tienen asignado un nivel de prioridad y se elige el cliente con mayor prioridad de la cola. En caso de haber varios con la misma probabilidad, se usa alguna de las disciplinas anteriores para elegir entre ellos.
    \end{itemize}

  \end{frame}

  \subsection{Notación de Kendall}
  \begin{frame}\frametitle{Notación de Kendall}
    \begin{exampleblock}{$$A/B/c/K/m/Z$$}
    \end{exampleblock}
    \begin{itemize}
    \item $A$ y $B$: describen $\tau$ y $S$ con los símbolos:
      \begin{itemize}
      \item [$G$]
        General: sólo se asume que $\tau_n$ o $S_n$ son independientes
      \item [$D$]
        Determínistica: $\tau$ es constante
      \item [$M$]
        De Markov: $\tau \sim exp(\lambda)$
        %% Estas dos no aparecen en en el apartado 2.1.1, ponerlas?
        %% \item [$E_k$] Erlang-k
        %% \item [$H_k$] k-stage hyperexponential
      \end{itemize}
    \item $c$: número de servidores
    \item $K$: capacidad máxima del sistema
    \item $m$: tamaño de la población
    \item $Z$: la disciplina de la cola (FIFO, LIFO, etc.)
    \end{itemize}

    Notación acortada (suponemos $K=\infty,m=\infty,Z=FIFO$):
    \begin{exampleblock}{$$A/B/c$$}
    \end{exampleblock}
  \end{frame}

  \subsection{Medidas de efectividad}
  \begin{frame}\frametitle{Medidas de efectividad}
    \begin{itemize}
    \item Aprovechamiento de los servidores: $\rho = \frac{\lambda}{c\mu}$
    \item Intensidad de tráfico: $a= W_S/E[\tau]$
    \item Tiempo medio de un cliente en el sistema: $W_W$
    \item Tiempo medio de un cliente en la cola: $W_Q$
    \item Percentil 90 del tiempo total en el sistema $W_W$: $\pi_{W} \lbrack 90 \rbrack$
    \item Percentil 90 del tiempo de espera en la cola $W_Q$: $\pi_{W} \lbrack 90 \rbrack$
    \item Número medio de clientes en el sistema: $L$
    \item Número medio de clientes en la cola: $L_Q$
    \item Probabilidad de que haya $n$ clientes en el sistema: $p_n$
    \end{itemize}
  \end{frame}

  \section{Leyes de Little}
  \begin{frame}\frametitle{Notación}
    Para $n \geq 1$, llamemos al cliente $n$-ésimo, el cual llega en el instante $t_n$, nombremos $W_n$ al tiempo 
    de espera del cliente con lo cual es inmediato deducir, que nuestro cliente abandonará la cola en el instante
    $t_n + W_n$, donde $0 = t_0 \leq t_1 \leq t_2 \leq... $.
    
    Es inmediato tambien que podemos definir una función indicadora que refleje si el $c_n$ está o no en el
    sistema:
    
    Indicador: $I_n(t) = \left\{\begin{array}{cc}
    1, & t_n \leq t <t_n + W_n\\
    0, & \textrm{en otro caso}
    \end{array}\right.$, de donde  $\int_{0}^{\infty} I_n(t)dt=W_n$.
    
    Sea $\Delta(t) = max \{i: t_n \leq t\}$ el número de llegadas en tiempo t. El número de clientes en el sistema
    viene dado por:
    
    \[N_t = \sum_{i=1}^{\Delta(t)} I_n(t)\]
  \end{frame}

  \begin{frame}
    A partir del tiempo de llegada ${t_n}$ y del tiempo de servicio ${W i}$ del cliente $i$-ésimo, podemos 
    construir ${N_t}$ paratodo $t$. Visualmente podemos entender de forma sencilla estos conceptos si trazamos 
    ${\Delta (t)}$ como una función escalonada y mostramos un tiempo de espera del cliente.

    \begin{figure}[H]
      \begin{center}
        \includegraphics[width=0.9\textwidth]{./imgs/fig1.png}
      \end{center}
    \end{figure}

  \end{frame}

  \begin{frame}
    A su vez podemos representar $N_t$ como una función escalonada (cuya información será obviamente menor que la primera figura) 
    representando únicamente el número de personas en el sistema.

    \begin{figure}[H]
      \begin{center}
        \includegraphics[width=0.9\textwidth]{./imgs/fig2.png}
      \end{center}
    \end{figure}
  \end{frame}

  \begin{frame}
  	Entonces definimos la media de estas cantidades a largo plazo como límites, cuando existan, denominándolos
  	como siguen:
  	
  	\begin{itemize}
  		\item $L = \lim_{T\to \infty}\frac{1}{T} \int_{0}^{T} N_t dt$
  		\item $W_W = \lim_{n\to \infty}\frac{1}{n} \sum_{i=1}^{n} W_i$
  		\item $\lambda = \lim_{t\to \infty}\frac{\Delta(t)}{t}$
  	\end{itemize}
  	
  	Donde conceptualmente estamos haciendo referencia a:
  	\begin{itemize}
  		\item $L$ es el número medio de clientes en el sistema,
  		\item $W_W$ es el tiempo medio de espera (de los clientes en el sistema), y
  		\item $\lambda$ es la tasa de llegada 
  	\end{itemize}
  \end{frame}
  
  \begin{frame}\frametitle{Teorema de Little}
    Si los límites $\lambda$ y $W_W$ existen y son finitos, entonces $L$ existe y es finito, donde 
    \[L = \lambda W_W\]
  \end{frame}
  \begin{frame}\frametitle{Idea de la demostración}
  	La idea intuitiva de este resultado es la relación que apúntabamos antes: para cualquier $T$ (donde ahora $N_T \geq 0$) escribimos:
  	\[ \frac{\int_{0}^{T} N_t dt}{T}=\frac{\sum_{i=1}^{\Delta (t)} W_i(t)}{T}-\frac{error}{T}=\frac{\Delta(t)}{T}\frac{\sum_{i=1}^{\Delta (T)} W_i(t)}{\Delta(T)}-\frac{error}{T}   \]
  	A medida que T se hace grande, la primera cantidad entre paréntesis a la derecha se aproxima a $\lambda$, la segunda a $\omega$, 
  	y el producto se acerca $\lambda\omega$. Por lo tanto, el teorema se obtendría si el término a la derecha 
  	se aproxima a 0 cuando $T\rightarrow\infty$. 
  \end{frame}
   \begin{frame}\frametitle{Ejemplo de uso}
   	Dados dos sistemas de colas, si los procesos que describen ambas son procesos estocásticos 
   	equivalentes entonces el tiempo de espera será el mismo. 
   \end{frame} 
  
  
  
\section{Modelos particulares}

  \subsection{Modelo D/D/1}
  \begin{frame}\frametitle{Caso $\lambda > \mu$}
    Intuitivamente, la cola será inestable y no podrá servir todas las peticiones. 

    Se tiene $a > 1$, $\rho > 1$.

    \begin{block}{Proposición}
      Notando $Q_n$ al tiempo de espera del $n$ ésimo cliente, tenemos que \[Q_1 = 0, \qquad Q_n = \left(\frac{1}{\mu} - \frac{1}{\lambda}\right)n, \quad \forall n\ge 2\]
    \end{block}

    \begin{block}{Corolario}
      El tiempo de espera total para el $n$ ésimo cliente es:
 
      \[W_1 = \frac{1}{\mu}, \qquad W_n = \left(\frac{1}{\mu} - \frac{1}{\lambda}\right)n + \frac{1}{\mu} \quad n\in \mathbb{N}\]
    \end{block}

    \begin{block}{Proposición}
      En un instante $t$ el número de clientes en el sistema será: 
      \[N_t = \left\{\begin{array}{lcc}
      0, && t < \frac{1}{\lambda}\\
      \lfloor t\lambda \rfloor - \left\lfloor\left(t-\frac{1}{\lambda}\right)\mu\right\rfloor, && \text{otro caso}
      \end{array}\right.\]
    \end{block}

  \end{frame}

  \begin{frame}\frametitle{Caso $\lambda \le \mu$}
    Se tiene $a < 1$, $\rho < 1$.

    \begin{block}{Proposición}
Se tiene $\forall n\in \mathbb{N}$:

\begin{align*}
Q_n = 0\\
W_n = \frac{1}{\mu}
\end{align*}
    \end{block}

    \begin{block}{Corolario}
 El tiempo de espera total para el $n$ ésimo cliente es únicamente el tiempo en ser servido.
    \end{block}

    \begin{block}{Proposición}
 En un instante $t$ el número de clientes en el sistema será: 
 \[N_t = \left\{\begin{array}{lcc}
          0, && t < \frac{1}{\lambda}\\
          0, && t \ge \frac{1}{\lambda}, t \in \left]\frac{n-1}{\lambda} + \frac{1}{\mu}, \frac{n-1}{\lambda} + \frac{1}{\lambda}\right[\\
          1, && \text{otro caso}
         \end{array}\right.\]
    \end{block}
  \end{frame}

  \subsection{Modelo M/M/1}
  \begin{frame}En este modelo consideramos un patrón de llegadas aleatorio y tiempo de servicio siguiendo una distribución 
  	exponencial.
  \end{frame}

\begin{frame}
la probabilidad de que se produzca una llegada 
en un intervalo de tiempo $h>0$ viene dada por:
\begin{align*}
& e^{-\lambda h}(\lambda h)= \left(1-\lambda h+\frac{(\lambda h)^2}{2!}- \dots\right) \lambda h = \\
& =\lambda h-(\lambda h)^2+\frac{(\lambda h)^3}{2!}-\dots+(-1)^n\frac{(\lambda h)^n}{(n-1)!} + \ldots = \lambda h+o(h)
\end{align*}
\end{frame}

\begin{frame}
	Por hipótesis la distribucion que sigue el tiempo de servicio viene dada por:
	\begin{equation*}
	W_S[t] = P[S\leq t] = 1-e^{-\mu t}, \quad t\ge 0.
	\end{equation*}
	
	Por tanto si un cliente está recibiendo el servicio, la probabilidad de que el servicio sea completado en un corto periodo de tiempo, $h$, viene dada por:
	\begin{equation*}
	1-e^{-\mu h} = 1 - \left(1-\mu h+\frac{(\mu h)^2}{2!}- \dots\right)=\mu h +o(h)
	\end{equation*}
\end{frame}
\begin{frame}
	Así, podemos deducir el diagrama de estado-transición para el sistema de colas M/M/1. 
	
	\begin{center}
	  \includegraphics[width=0.9\textwidth]{./imgs/diagrama.png}
	\end{center}
	
	Además dado que $\lambda/\mu=\rho$ utilizando la serie calculada en (\ref{eq:stabseries}) nos queda que:
	\begin{equation*}
	Z=1+\rho+\rho^2+\dots+\rho^n+\dots=\frac{1}{(1-\rho)}
	\end{equation*}
\end{frame}
\begin{frame}
	Para sumar la serie de Z hemos usado la suma de la serie geométrica ampliamente conocida.
	Así por la definicion dada en \eqref{p0} y \eqref{pn}
	
	\begin{equation}
	p_n=P[N=n]=(1-\rho)\rho^n, n=0,1,2...
	\label{eq:p_n}
	\end{equation}
\end{frame}
\begin{frame}
	Podemos observar que \eqref{eq:p_n} es la función masa de probabilidad de una variable aleatoria geométrica, esto
	es, $N$ sigue una distribucion geométrica con $p = 1-\rho$ y $q = \rho$. Por tanto podemos usar para obtener $L$:
	\begin{equation*}
	L=E[N]=q/p=\frac{\rho}{(1-\rho)}
	\end{equation*}
	
	Usando la Ley de Little:
	\begin{equation*}
	W_W=E[W]=\frac{L}{\lambda}=\frac{W_s}{(1-\rho)}
	\end{equation*}
	Podemos también calcular utilizando la probabilidad de que el servidor esté ocupado en estado de equilibrio:
	\begin{equation*}
	P[\textbf{Servidor ocupado}]=1-p_0=1-(1-\rho)=\rho
	\end{equation*}
\end{frame}


  \subsection{Paradoja del tiempo de espera}
  \begin{frame}\frametitle{Paradoja del tiempo de espera}
    El autobús pasa cada $\alpha$ minutos por la parada, y llegamos en un momento aleatorio. ¿Cuánto tiempo esperamos que tarde en llegar el autobús?

    Intuición:  $\frac{\alpha}{2} $
    \pause
    \begin{alertblock}{FALSO}
    \end{alertblock}

    Respuesta correcta: depende de la varianza del tiempo entre llegadas
    \begin{itemize}
    \item Varianza 0: $\frac{\alpha}{2}$
    \item Varianza infinita: tiempo infinito
    \end{itemize}
  \end{frame}

%% Bibliografía
\section {Referencias}
\begin{frame}
\frametitle{Referencias}
\footnotesize{
  \begin{thebibliography}{99} % Beamer does not support BibTeX so references must be inserted manually as below
    \bibitem[Allen, 1990]{allen} Arnold O.Allen
      \newblock Probability, Statistics and Queueing Theory with Computer Science Applications\\
      \newblock \emph{Academic Press}

    \bibitem[Gross, 2008]{gross} Donald Gross, John F.Shortle, James M.Thompson, Carl M.Harris
      \newblock Fundamentals of Queueing Theory\\
      \newblock \emph{Wiley}

    \bibitem[Gunavathi, 2010]{gunavathi} P.Kandasamy, K.Thilagavathi, K.Gunavathi
      \newblock Probability and Queueing Theory (2010)\\
      \newblock \emph{S. Chand \& Company}

    \bibitem[Wolff, 2011]{wolff} Ronald W. Wolff
      \newblock Little's law and Related Results\\
      \newblock \emph{University of California at Berkeley}

    \bibitem[Takács, 1962]{takacs} Lajos Takákcs
      \newblock Introduction to the Theory of Queues\\
      \newblock \emph{Oxford University Press}

  \end{thebibliography}
}
\end{frame}
\end{document}
