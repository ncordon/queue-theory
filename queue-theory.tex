\documentclass[a4paper,10pt]{scrartcl}

% Inclusión de paquetes
\usepackage[utf8]{inputenc}
\usepackage{amsmath}
\usepackage{amssymb}
\usepackage{enumerate}
\usepackage{verbatim}
\usepackage{amsthm}

% Definiciones
\newtheorem*{mydef}{Definición}
\newtheorem{mydefn}{Definición}
\newtheorem{theorem}{Teorema}
\everymath{\displaystyle} % Displaystyle por defecto

% Comandos
\newcommand{\Referencia}[4]{\indent #1, \textbf{#2}. \textit{#3}, \textit{#4}.\\}
\renewcommand\refname{Referencias}
\renewcommand\contentsname{Contenidos}
\numberwithin{equation}{section}
\setlength{\parindent}{0cm} % Sin sangrías
\setlength{\parskip}{0.1cm}

% Sintaxis: \Algoritmo{Elementos de entrada}{Elementos de proceso}{Elementos de salida}
\newcommand{\Algoritmo}[3]{\textbf{Entrada} \begin{itemize} #1 \end{itemize} \textbf{Proceso} \begin{enumerate} #2 \end{enumerate} \textbf{Salida} \begin{itemize} #3 \end{itemize}}


\title{Teoría de Colas}
\author{
  Marta Andrés\and
  Ignacio Cordón\and
  Bartolomé Ortiz\and
}
\date{}


\begin{document}
\maketitle
\tableofcontents
\pagebreak
\section{Definición y aplicaciones}


\subsection{Definiciones}
% Incluyo las definiciones previas y la definición de matriz tridiagonal como inicio.
Ejemplo de cita: \cite{Ciarlet}

\newpage
\begin{thebibliography}{10}
\expandafter\ifx\csname url\endcsname\relax
  \def\url#1{\texttt{#1}}\fi
\expandafter\ifx\csname urlprefix\endcsname\relax\def\urlprefix{URL }\fi
\expandafter\ifx\csname href\endcsname\relax
  \def\href#1#2{#2} \def\path#1{#1}\fi

\bibitem{Ciarlet}
Ciarlet, P. G. (1982).
  Introduction to numerical linear algebra and optimisation.
  Cambridge University Press

\end{thebibliography}

\end{document}